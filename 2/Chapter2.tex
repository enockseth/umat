% this file is called up by thesis.tex
% content in this file will be fed into the main document

\chapter{Literature Review} % top level followed by section, subsection


% ----------------------- contents from here ------------------------

\section{Background Study}
Transportation is a system or means of transporting people or goods from one place to another. Transportation is key to the movement of people, goods and services leading to development. In Sub-Sahara and most African countries, the major means of transportation is by road transport.

Transportation in Ghana dates back to precolonial era. Most major roads and railway lines were constructed by the Europeans to facilitate the movement of natural resources and raw materials such as timber, gold and bauxite to generating communities to the coastal areas. Transport in all its modes is one of the most important catalysts for development \textit{(cite)}. %http://www.mrh.gov.gh/files/publications/National_Transport_Policy___December__2008.pdf
After independence road transportation remains a major player in Ghana's economy connecting all ten administrative regions of Ghana. 

\section{Categories of Road Transport}
Road Transport in Ghana can be grouped into 4 main categories:\textit{(cite)}
\begin{itemize}
	\item Urban
	\item Express Services
	\item Rural - Urban
	\item Rural
\end{itemize}

\subsection{Urban}
Urban is used mostly by residents of urban areas commuting to and from work, school and other places of convenience. It is usually carried out by private taxis, personal cars, \textit{trotros} and state owned transport services.

\subsection{Express Services}
Express services are demanded services by travelers; this could be in the form of taxis, buses and minivans.

\subsection{Rural-Urban}
Rural-Urban transportation is by far the most popular group of public transportation in Ghana as well as a major factor in Rural-Urban migration. Market trading in the urban centres of Ghana is a predominantly female economic activity and is a fundamental element in the survival strategies of many low-income households. Petty trading is the predominant form of commercial activity and as such, given the financial constraints inherent in this form of trading, necessitates regular travel between wholesale markets and selling places on the part of female traders. Female traders make use of the public transport system, in combination with supplementary services such as portering, to meet their travel needs \citep{grieco1995informal}. 

\subsection{Rural}
This is transportation mainly within rural areas. Depending on the status of each rural area transportation could range for donkey pulled carts, foot, bicycles and motorcycles, taxis and minivans in average conditions.

\subsection{Owners or Operators}
Transportation service operators or owners can be classified as:
\begin{itemize}
	\item Private
	\item Government
	\item Private-Government
\end{itemize}

\section{The Industry Structure}
Within the main metropolitan areas of Ghana, there are two main forms of public transport operations \citep{finn2009new}:
\begin{enumerate}
	\item \textit{Tro-tro} (mini buses) and shared taxi services, which are managed by unions and cooperatives and offer services along defined routes, usually between terminals or ‘lorry parks/stations’. These operations suffer from a number of quality problems including:
	
	\begin{enumerate}
		\item operation of a ‘fill and go’ system which can result in long delays for users in the offpeak, and difficulty to board along the route;
		\item large numbers of vehicles parked at
		terminals in the off-peak leading to congestion, inefficiency, and long hours for
		drivers; 
		\item lack of incentives for vehicle owners to improve their vehicles or to train their drivers properly.
	\end{enumerate}
	
	\item Large bus services, mostly provided by the new Metro Mass Transit (MMT), a quasi-private company that receives favorable financial support from the government.
\end{enumerate}

\subsection{Operating unions and associations}. 
Individually or privately operated transport services are members of unions or associations. These unions and associations serve as regulatory and mouth-piece to each of their members \citep{fouracre1994public}. There are three major operating unions and associations.

\subsection{Ghana Private Road Transport Union}
The Ghana Private Road Transport Union (GPRTU), a national union, is reported to have about 90\% of the \textit{tro-tro} and shared taxi business. The fundamental units are Locals, which operate the individual routes, and Branches, which are regional clusters of Locals. GPRTU represents the interests of both drivers and of vehicle owners.

\subsection{Progressive Transport Owners Association}
The Progressive Transport Owners Association (PROTOA), is a national association that operates both tro-tro and shared taxi business and is organized along the same structure as the GPRTU. PROTOA mainly represents the interest of owners. 

\subsection{Ghana Co-operative Transport Association}
The Ghana Co-operative Transport Association (GCTA) is a national association also organized along the lines of GPRTU and represents interest of both owners and drivers.

\subsection{Other private operators}
Other private operators, such as Agate, Kingdom Transport, and Pergah Transport, are companies operating several buses and offering a range of services including contract service, urban services, and intercity services.

\subsection{Ghana Road Transport Coordination Council}
Ghana Road Transport Coordination Council (GRTCC) is an umbrella body of all transport operators in Ghana, including the unions and associations, other locally based associations, and other operators (both passenger and road haulers). GRTCC represents the interests of road transport operators, especially in negotiating with the Government of Ghana for transport tariffs and assistance in acquisition of buses.

\section{Operators}
Transport Operators are individual, state owned or both private and government partnerships managing the affairs of a particular brand.

\subsection{Intercity STC Limited}
Intercity bus transport is a popular means of travelling between cities and aligning villages and towns in Ghana. Its services include freight and passenger movements from one location to the other. For this service to be provided, a company has to be formed. As a result of that there have been concerted attempts by various past Governments of Ghana to offer intercity bus transport service to her citizen. One of such efforts is the establishment of Intercity State Transport Company (ISTC). But there is a number of private transport operators 
of which Ghana Private Road Transport Union (GPRTU) offers about $70-80\%$ of passenger and freight traffic. This is an off shoot of intra urban dominance of GPRTU of $70-80\%$ \citep{abane2011travel}. GPRTU has been able to co-opt other intercity bus transport operators by sharing some of its terminals/stations with other transport 
companies/union such as VVIP/VIP, DIPLOMAT. Aside this, some private owners or operators like VIP/VVIP, DIPLOMAT for instance are either members or former members/executives of the union. Other unions/transport operators in the industry are Concerned Drivers Union, Progressive Transport Owners Association and Co-operative \citep{ojobus}.

\textit{Insert map of STC Terminals across Ghana}

\subsection{Metro Mass Transit}
Metro  Mass  Transit  Limited, Ghana was established in 2001 by the former President of Ghana, John Kuffour who directed the re-introduction of public mass transport in the metropolitan and municipal areas to ensure safe, affordable, efficient and reliable movement of Ghanaians. Since then, the Government has been actively promoting public mass transportation \citep{olateju2009appraisal}. MMT receives financial support from the Government and currently operates about 500 buses of which some 200 operate in the greater Accra area \citep{finn2009new}.

\subsection{Aayalolo}
Aayalolo Bus Rapid Transit(BRT) system has recently been incorporated. Since November 2016, the company has been running three services on the Amasaman-Tudu/
Accra Central corridor. There are plans to roll out the Adenta-Tudu/Accra Central corridor and subsequently, another service along the Tema Beach Road-Tudu/Accra Central corridor. There are plans to replicate the BRT mass transit services in other major Ghanaian cities. Prior to the Aayalolo bus service \citep{agyemang2017mode}.

\section{Maps}
A map is a graphic representation or scale model of spatial concepts. It is a means for conveying geographic information. Maps are a universal medium for communication, easily appreciated and understood by most people, regardless of language or culture. Maps record the geographical information that is fundamental to reconstructing past places, towns, even cities.

\section{Digital mapping}
Digital mapping is the process by which a collection of data is compiled and converted into a virtual image. The primary function of this technology is to produce maps that give accurate representation of a particular area, detailing major road arteries and other points of interest. The technology also allows the calculation of distances from one place to another.
The roots of digital mapping lie within traditional paper maps. Paper maps provide basic landscapes similar to digitized road maps, yet are often cumbersome, cover only a designated area, lack many specific details such as road blocks etc. In addition, there is no way to update a paper map except to obtain a new version. Conversely, digital maps, in many cases, can be updated.
Early digital maps had the same functionality as traditional maps-that is, they provided a ‘virtual view’ of roads generally outlined by the terrain encompassing the mounding area. However, as digital maps have grown with the expansion of GPS technology in the past decade, live traffic updates, points of interests and service locations have been added to enhance digital maps to be more ‘user conscious’ \textit{cite}. Digital maps heavily rely on a vast amount of data collection over time.

\section{Existing Technologies}
Digital maps have changed the perception of maps and introduced much flexibility compared to paper maps. Existing technologies such as OpenStreetMap, Google Maps, Bing Maps and Taximap provides web map services and that captures transportation information but not into much detail. Taximap on the other seeks to localize this \citep{vinet2014arch}.

\subsection{OpenStreetMap}
OpenstreetMap (OSM) is a collaborative project started in England in 2004 by Steve Coast. The aim of OSM is to create and provide free geographic data. The project aims to compensate the lack of free data because geographic data, even freely available, are provided with licenses restricting the use of information and the creativity according to project leaders. The data are distributed under the license “Creative Commons Attribution-ShareAlike 2.0 license”. This license allows using the data completely freely, in condition to distribute any derived data under the same license. For instance, corrected OSM data cannot be sold.
Data stored in OSM by contributors of the project are modelled and stored in tagged geometric primitives. For example, a road is a polyline with tags \textit{highway= “primary”, oneway= “no” and name= “N10”}. Geometric primitives are of three types: points, paths (polylines) and relationships (linking points and paths with tags) that are not really geometric primitives. The surfaces are represented by close paths. Data are available from any area specified for export in a specific XML based format. It has to be translated if anyone wishes to use the data in another application. Data is captured using GIS software adapted to OSM data with editing functions to create OSM geometric primitives and tag them. Different software exists to edit and capture OSM data (Potlatch, JOSM, Merkaartor). OSM applications currently aim to foster mapping creativity of potential contributors of geographic data \citep{girres_quality_2010}.

\subsection{Google Maps}
Google Maps is a propitiatory tool for navigation. Google Maps is used by many people around the world. Google has an online database of structures; which is better covered in most developed countries compared to developing countries. Searching for places of interests is done by geocoding - \textit{converting names or addresses of places to a location on a map} or reverse geocoding - \textit{converting latitudes and longitudes to a readable address or name}. Google Maps is made possible through Volunteered Geographic Information (VGI) and other third party proprietary data sources. Adding, modifying and deleting features from Google Maps has recently become difficult for a new contributor. Google Maps like OpenStreetMap could also fail to pin point the exact location of a terminal.

% Clound route from one location to another but relative to position of towns not the various bus terminals

\subsection{Taximap}
Taximap is a social enterprise start-up and public transport search portal in South Africa. It provides information about minibus taxi routes, fares and operating hours with commuters. Information about on over 800 mapped routes are available. The platform rely on user feedback to keep the routes up-to-date and accurate, by encouraging users to leave feedbacks. Taximaps does well by telling the exact Taximap has one drawback by not allowing users of the platform to specify both their departure and destination locations. 


\section{Proposed System}

\section{Software Review}

\subsection{Arch Linux}
The Arch Linux operating system was most suitable for my project. Arch Linux is a native Linux based operating system produced for computers with chipsets based on i686 and x86-64 architectures (Griffin, 2002); it was chosen because of its simplicity, community involvement and a well documented wiki and community support \citep{vinet2014arch}.

\subsection{Python}
Python is an object oriented, interpreted programming language. It runs on a wide variety of systems which include Linux, Unix, Mac OS, Windows, BSD, etc. It also has several implementations, such as IronPython which runs on .NET CLR, Jython, amongst others \citep{van2007python}. 

\subsection{Django}
Django is a popular Python Web Development Framework. It is dubbed the web develoment framework from developers with deadline. Django was born from the newsroom as a framework for development new and media related web applications.

\subsection{Emacs}
Emacs was the development environment used to develop this project. It was initially released in 1976 and active development continues to date. Its flexibility is one of its major advantages. Amongst other things, it provides a text editor with efficient key bindings and a method for interacting with inferior shells such as Python shell. What this means is that, after a line code has been written, it can be directly sent to the Python interpreter, where the code is executed and the results are shown immediately. This makes exploratory development much more efficient and streamlined. 

\subsection{Java OpenStreetMap}
Java OpenStreetMap (JOSM) is an extensible editor for OpenStreetMap. It supports loading GPX tracks, background imagery and OSM data from local sources as well as from online sources and allows to edit the OSM data (nodes, ways, and relations) and their metadata tags. JOSM is an open source. JOSM was used for the initial processing of data collected and uploading into OpenStreetMap.

\subsection{PostgresSQL}
The database management system which is more suitable for my proposed system is PostgreSQL. PostgreSQL is an open source relational database management system that began as a University of California, Berkeley project. PostgreSQL was selected above MySQL and MSSQL (Microsoft SQL Server) because it's open source and great support for extensions. PostGIS extension provides a great support for geographic data. It also powers OpenStreetMap and Skype databases.

It has enterprise class features such as SQL windowing functions, the ability to create aggregate functions and also utilize them in window constructs, common table and
recursive common table expressions, and streaming replication. These features are rarely found in other open source database platforms, but commonly found in newer versions of the proprietary databases such as Oracle, SQL Server, and IBM DB2. What sets it apart from other databases, including the proprietary ones we just mentioned,
is the ease with which you can extend it without changing the underlying base—and in many cases, without any code compilation. Not only does it have advanced features,
but it performs them quickly. It can outperform many other databases, including proprietary ones for many types of database workloads.




% ---------------------------------------------------------------------------
% ----------------------- end of thesis sub-document ------------------------
% ---------------------------------------------------------------------------
