
% this file is called up by thesis.tex
% content in this file will be fed into the main document

%: ----------------------- introduction file header -----------------------
\chapter{INTRODUCTION}

% the code below specifies where the figures are stored
\ifpdf
    \graphicspath{{1_introduction/figures/PNG/}{1_introduction/figures/PDF/}{1_introduction/figures/}}
\else
    \graphicspath{{1_introduction/figures/EPS/}{1_introduction/figures/}}
\fi

% ----------------------------------------------------------------------
%: ----------------------- introduction content ----------------------- 
% ----------------------------------------------------------------------



%: ----------------------- HELP: latex document organisation
% the commands below help you to subdivide and organise your thesis
%    \chapter{}       = level 1, top level
%    \section{}       = level 2
%    \subsection{}    = level 3
%    \subsubsection{} = level 4
% note that everything after the percentage sign is hidden from output
This chapter presents a statement of the problem tackled in this project work, with specific focus on the motivation for conducting the project. Further the chapter highlights the objectives of the project, expected outcomes as well as methods employed for the accomplishment of the objectives. Finally, the tools and facilities used, the scope and organization of the project work are discussed. 

\section{Problem Statement}
Road transport service provided by both formal and informal sectors in Ghana is by far the most popular and principal means of conveying passengers, goods and other services between any two or more locations \citep{aidoo_passengers_2013}. Some reports have it that, road transport accounts for over $ 95\% $ of all passenger and freight traffic and about $ 97\% $ of all passenger miles in Ghana \citep[p.~195]{unesco_transportation:_????}, a country that is experiencing rapid demographic and economic growth. The vast majority of passengers commuting between places, be it \textit{intra-city} or \textit{inter-city}, mostly rely on public road transport services in the form of privately owned or corporate taxis, \textit{tro tros} (shared minivans), buses commuting between major cities.

In spite of the heavy reliance on public road transport services by the general populace in Ghana, finding transport terminals which offer reliable road transport services is not as easy as it should be. The difficulty in finding transport terminals is attributable to the fact that little or no information about the availability of transport services and their locations is accessible to the public. Additionally, the non-existent of a means to compare transport fares by various service providers often makes it difficult for the potential passenger to make the right choices. It is the goal of every potential passenger to find the fastest, safest and most cost-efficient means of transiting from one location to another. 

Inspired by the aforementioned shortcomings of the existing public road transportation system, this project seeks to develop road transport terminal search tool
aimed at mitigating, if not eliminate entirely, these problems with the public road transport industry.

%\newpage
\section{Project Objectives}
The specific objectives of this project include the following:
\begin{itemize}
	\item To develop a web application that provides detailed information about transport terminals in Ghana to mitigate the difficulty in finding transport terminals and also to provide a platform for to compare other factors by travelers.
		\item To provide reusable data that can be accessed in mobile and desktop applications that support Geo Uniform Resource Identifier (URI) scheme.
		\item With Global Positioning System (GPS) supported devices, the work also provides a means of navigating to destination based by incorporating existing mapping solution such as
	    \begin{itemize}[label=$\circ$]
		\item OSMAnd
		\item MAPS.ME
		\item Google Maps
		\item Apple Maps
		\item Marble
		\item Google Earth
		\end{itemize}
\end{itemize}

\section{Project Outcomes}
The following will be achieved at the end of the project:
\begin{itemize}
	\item An web application;
%	\item \sout{Possibly generate high definition printer friendly maps that serves nearly same purpose as the web platform.}
	\item A detail map of selected transport terminals on OpenStreetMap.
\end{itemize}


\section{Methods Used}
The methods to be used for the project are as below:
\begin{itemize}
	\item Literature review on proposed topic
	\item Study and understanding of online maps (creating, updating, deleting);
	\item The system will be developed on a handful of local machines, but with scalability and ease of deployment on any kind of infrastructure;
	\item The system will be prototyped in Python programming language and Django web framework. If this language proves good enough for deployment purposes it will be used in the final product;
	\item Survey and crowd-sourcing information on some transport terminals to facilitate database creation;
	\item QGIS will be used to clean and analyze data collected;
	\item Spiral software development model.
\end{itemize}

\section{Tools and Facilities Used}
The facilities required for this project include:
\begin{itemize}
	\item University of Mines and Technology (UMaT) library;
	\item Internet;
	\item General search engines as such Wikipedia and DuckDuckGo;
	\item Open Source software repositories such as GitHub;
	\item OpenStreetMap;
	\item Documentation of any software or libraries used.
\end{itemize}

\section{Scope of Work}
This works seeks to aid travelers acquire detailed information on selected bus terminals in Ghana. This is to enable better trip planning, reduce time taken finding these terminals and also allowing passengers to compare transport fares and choosing the best option they can afford.
Further more the application only indicates source and destination terminals and works in the any modern web browser such as Mozilla Firefox or Google Chrome.

\section{Organization of Project}
This project is divided into five chapters. The first chapter talks about the problem to be solved, the objectives, the methods, tools and facilities used, and project outcomes. The second chapter discusses relevant literature and related works. The third chapter discusses how the problem was solved. The fourth chapter talks about the operation of the developed application. The project concludes in the fifth chapter where, limitations and recommendations for future improvements were discussed.

